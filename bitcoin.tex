
%% bare_jrnl_compsoc.tex
%% V1.4a
%% 2014/09/17
%% by Michael Shell
%% See:
%% http://www.michaelshell.org/
%% for current contact information.
%%
%% This is a skeleton file demonstrating the use of IEEEtran.cls
%% (requires IEEEtran.cls version 1.8a or later) with an IEEE
%% Computer Society journal paper.
%%
%% Support sites:
%% http://www.michaelshell.org/tex/ieeetran/
%% http://www.ctan.org/tex-archive/macros/latex/contrib/IEEEtran/
%% and
%% http://www.ieee.org/

%%*************************************************************************
%% Legal Notice:
%% This code is offered as-is without any warranty either expressed or
%% implied; without even the implied warranty of MERCHANTABILITY or
%% FITNESS FOR A PARTICULAR PURPOSE! 
%% User assumes all risk.
%% In no event shall IEEE or any contributor to this code be liable for
%% any damages or losses, including, but not limited to, incidental,
%% consequential, or any other damages, resulting from the use or misuse
%% of any information contained here.
%%
%% All comments are the opinions of their respective authors and are not
%% necessarily endorsed by the IEEE.
%%
%% This work is distributed under the LaTeX Project Public License (LPPL)
%% ( http://www.latex-project.org/ ) version 1.3, and may be freely used,
%% distributed and modified. A copy of the LPPL, version 1.3, is included
%% in the base LaTeX documentation of all distributions of LaTeX released
%% 2003/12/01 or later.
%% Retain all contribution notices and credits.
%% ** Modified files should be clearly indicated as such, including  **
%% ** renaming them and changing author support contact information. **
%%
%% File list of work: IEEEtran.cls, IEEEtran_HOWTO.pdf, bare_adv.tex,
%%                    bare_conf.tex, bare_jrnl.tex, bare_conf_compsoc.tex,
%%                    bare_jrnl_compsoc.tex, bare_jrnl_transmag.tex
%%*************************************************************************


% *** Authors should verify (and, if needed, correct) their LaTeX system  ***
% *** with the testflow diagnostic prior to trusting their LaTeX platform ***
% *** with production work. IEEE's font choices and paper sizes can       ***
% *** trigger bugs that do not appear when using other class files.       ***                          ***
% The testflow support page is at:
% http://www.michaelshell.org/tex/testflow/


\documentclass[10pt,journal,compsoc]{IEEEtran}
%
% If IEEEtran.cls has not been installed into the LaTeX system files,
% manually specify the path to it like:
% \documentclass[10pt,journal,compsoc]{../sty/IEEEtran}





% Some very useful LaTeX packages include:
% (uncomment the ones you want to load)


% *** MISC UTILITY PACKAGES ***
%
%\usepackage{ifpdf}
% Heiko Oberdiek's ifpdf.sty is very useful if you need conditional
% compilation based on whether the output is pdf or dvi.
% usage:
% \ifpdf
%   % pdf code
% \else
%   % dvi code
% \fi
% The latest version of ifpdf.sty can be obtained from:
% http://www.ctan.org/tex-archive/macros/latex/contrib/oberdiek/
% Also, note that IEEEtran.cls V1.7 and later provides a builtin
% \ifCLASSINFOpdf conditional that works the same way.
% When switching from latex to pdflatex and vice-versa, the compiler may
% have to be run twice to clear warning/error messages.






% *** CITATION PACKAGES ***
%
\ifCLASSOPTIONcompsoc
  % IEEE Computer Society needs nocompress option
  % requires cite.sty v4.0 or later (November 2003)
  \usepackage[nocompress]{cite}
\else
  % normal IEEE
  \usepackage{cite}
\fi
% cite.sty was written by Donald Arseneau
% V1.6 and later of IEEEtran pre-defines the format of the cite.sty package
% \cite{} output to follow that of IEEE. Loading the cite package will
% result in citation numbers being automatically sorted and properly
% "compressed/ranged". e.g., [1], [9], [2], [7], [5], [6] without using
% cite.sty will become [1], [2], [5]--[7], [9] using cite.sty. cite.sty's
% \cite will automatically add leading space, if needed. Use cite.sty's
% noadjust option (cite.sty V3.8 and later) if you want to turn this off
% such as if a citation ever needs to be enclosed in parenthesis.
% cite.sty is already installed on most LaTeX systems. Be sure and use
% version 5.0 (2009-03-20) and later if using hyperref.sty.
% The latest version can be obtained at:
% http://www.ctan.org/tex-archive/macros/latex/contrib/cite/
% The documentation is contained in the cite.sty file itself.
%
% Note that some packages require special options to format as the Computer
% Society requires. In particular, Computer Society  papers do not use
% compressed citation ranges as is done in typical IEEE papers
% (e.g., [1]-[4]). Instead, they list every citation separately in order
% (e.g., [1], [2], [3], [4]). To get the latter we need to load the cite
% package with the nocompress option which is supported by cite.sty v4.0
% and later. Note also the use of a CLASSOPTION conditional provided by
% IEEEtran.cls V1.7 and later.





% *** GRAPHICS RELATED PACKAGES ***
%
\ifCLASSINFOpdf
  % \usepackage[pdftex]{graphicx}
  % declare the path(s) where your graphic files are
  % \graphicspath{{../pdf/}{../jpeg/}}
  % and their extensions so you won't have to specify these with
  % every instance of \includegraphics
  % \DeclareGraphicsExtensions{.pdf,.jpeg,.png}
\else
  % or other class option (dvipsone, dvipdf, if not using dvips). graphicx
  % will default to the driver specified in the system graphics.cfg if no
  % driver is specified.
  % \usepackage[dvips]{graphicx}
  % declare the path(s) where your graphic files are
  % \graphicspath{{../eps/}}
  % and their extensions so you won't have to specify these with
  % every instance of \includegraphics
  % \DeclareGraphicsExtensions{.eps}
\fi
% graphicx was written by David Carlisle and Sebastian Rahtz. It is
% required if you want graphics, photos, etc. graphicx.sty is already
% installed on most LaTeX systems. The latest version and documentation
% can be obtained at: 
% http://www.ctan.org/tex-archive/macros/latex/required/graphics/
% Another good source of documentation is "Using Imported Graphics in
% LaTeX2e" by Keith Reckdahl which can be found at:
% http://www.ctan.org/tex-archive/info/epslatex/
%
% latex, and pdflatex in dvi mode, support graphics in encapsulated
% postscript (.eps) format. pdflatex in pdf mode supports graphics
% in .pdf, .jpeg, .png and .mps (metapost) formats. Users should ensure
% that all non-photo figures use a vector format (.eps, .pdf, .mps) and
% not a bitmapped formats (.jpeg, .png). IEEE frowns on bitmapped formats
% which can result in "jaggedy"/blurry rendering of lines and letters as
% well as large increases in file sizes.
%
% You can find documentation about the pdfTeX application at:
% http://www.tug.org/applications/pdftex






% *** MATH PACKAGES ***
%
%\usepackage[cmex10]{amsmath}
% A popular package from the American Mathematical Society that provides
% many useful and powerful commands for dealing with mathematics. If using
% it, be sure to load this package with the cmex10 option to ensure that
% only type 1 fonts will utilized at all point sizes. Without this option,
% it is possible that some math symbols, particularly those within
% footnotes, will be rendered in bitmap form which will result in a
% document that can not be IEEE Xplore compliant!
%
% Also, note that the amsmath package sets \interdisplaylinepenalty to 10000
% thus preventing page breaks from occurring within multiline equations. Use:
%\interdisplaylinepenalty=2500
% after loading amsmath to restore such page breaks as IEEEtran.cls normally
% does. amsmath.sty is already installed on most LaTeX systems. The latest
% version and documentation can be obtained at:
% http://www.ctan.org/tex-archive/macros/latex/required/amslatex/math/





% *** SPECIALIZED LIST PACKAGES ***
%
%\usepackage{algorithmic}
% algorithmic.sty was written by Peter Williams and Rogerio Brito.
% This package provides an algorithmic environment fo describing algorithms.
% You can use the algorithmic environment in-text or within a figure
% environment to provide for a floating algorithm. Do NOT use the algorithm
% floating environment provided by algorithm.sty (by the same authors) or
% algorithm2e.sty (by Christophe Fiorio) as IEEE does not use dedicated
% algorithm float types and packages that provide these will not provide
% correct IEEE style captions. The latest version and documentation of
% algorithmic.sty can be obtained at:
% http://www.ctan.org/tex-archive/macros/latex/contrib/algorithms/
% There is also a support site at:
% http://algorithms.berlios.de/index.html
% Also of interest may be the (relatively newer and more customizable)
% algorithmicx.sty package by Szasz Janos:
% http://www.ctan.org/tex-archive/macros/latex/contrib/algorithmicx/




% *** ALIGNMENT PACKAGES ***
%
%\usepackage{array}
% Frank Mittelbach's and David Carlisle's array.sty patches and improves
% the standard LaTeX2e array and tabular environments to provide better
% appearance and additional user controls. As the default LaTeX2e table
% generation code is lacking to the point of almost being broken with
% respect to the quality of the end results, all users are strongly
% advised to use an enhanced (at the very least that provided by array.sty)
% set of table tools. array.sty is already installed on most systems. The
% latest version and documentation can be obtained at:
% http://www.ctan.org/tex-archive/macros/latex/required/tools/


% IEEEtran contains the IEEEeqnarray family of commands that can be used to
% generate multiline equations as well as matrices, tables, etc., of high
% quality.




% *** SUBFIGURE PACKAGES ***
%\ifCLASSOPTIONcompsoc
%  \usepackage[caption=false,font=footnotesize,labelfont=sf,textfont=sf]{subfig}
%\else
%  \usepackage[caption=false,font=footnotesize]{subfig}
%\fi
% subfig.sty, written by Steven Douglas Cochran, is the modern replacement
% for subfigure.sty, the latter of which is no longer maintained and is
% incompatible with some LaTeX packages including fixltx2e. However,
% subfig.sty requires and automatically loads Axel Sommerfeldt's caption.sty
% which will override IEEEtran.cls' handling of captions and this will result
% in non-IEEE style figure/table captions. To prevent this problem, be sure
% and invoke subfig.sty's "caption=false" package option (available since
% subfig.sty version 1.3, 2005/06/28) as this is will preserve IEEEtran.cls
% handling of captions.
% Note that the Computer Society format requires a sans serif font rather
% than the serif font used in traditional IEEE formatting and thus the need
% to invoke different subfig.sty package options depending on whether
% compsoc mode has been enabled.
%
% The latest version and documentation of subfig.sty can be obtained at:
% http://www.ctan.org/tex-archive/macros/latex/contrib/subfig/




% *** FLOAT PACKAGES ***
%
%\usepackage{fixltx2e}
% fixltx2e, the successor to the earlier fix2col.sty, was written by
% Frank Mittelbach and David Carlisle. This package corrects a few problems
% in the LaTeX2e kernel, the most notable of which is that in current
% LaTeX2e releases, the ordering of single and double column floats is not
% guaranteed to be preserved. Thus, an unpatched LaTeX2e can allow a
% single column figure to be placed prior to an earlier double column
% figure. The latest version and documentation can be found at:
% http://www.ctan.org/tex-archive/macros/latex/base/


%\usepackage{stfloats}
% stfloats.sty was written by Sigitas Tolusis. This package gives LaTeX2e
% the ability to do double column floats at the bottom of the page as well
% as the top. (e.g., "\begin{figure*}[!b]" is not normally possible in
% LaTeX2e). It also provides a command:
%\fnbelowfloat
% to enable the placement of footnotes below bottom floats (the standard
% LaTeX2e kernel puts them above bottom floats). This is an invasive package
% which rewrites many portions of the LaTeX2e float routines. It may not work
% with other packages that modify the LaTeX2e float routines. The latest
% version and documentation can be obtained at:
% http://www.ctan.org/tex-archive/macros/latex/contrib/sttools/
% Do not use the stfloats baselinefloat ability as IEEE does not allow
% \baselineskip to stretch. Authors submitting work to the IEEE should note
% that IEEE rarely uses double column equations and that authors should try
% to avoid such use. Do not be tempted to use the cuted.sty or midfloat.sty
% packages (also by Sigitas Tolusis) as IEEE does not format its papers in
% such ways.
% Do not attempt to use stfloats with fixltx2e as they are incompatible.
% Instead, use Morten Hogholm'a dblfloatfix which combines the features
% of both fixltx2e and stfloats:
%
% \usepackage{dblfloatfix}
% The latest version can be found at:
% http://www.ctan.org/tex-archive/macros/latex/contrib/dblfloatfix/




%\ifCLASSOPTIONcaptionsoff
%  \usepackage[nomarkers]{endfloat}
% \let\MYoriglatexcaption\caption
% \renewcommand{\caption}[2][\relax]{\MYoriglatexcaption[#2]{#2}}
%\fi
% endfloat.sty was written by James Darrell McCauley, Jeff Goldberg and 
% Axel Sommerfeldt. This package may be useful when used in conjunction with 
% IEEEtran.cls'  captionsoff option. Some IEEE journals/societies require that
% submissions have lists of figures/tables at the end of the paper and that
% figures/tables without any captions are placed on a page by themselves at
% the end of the document. If needed, the draftcls IEEEtran class option or
% \CLASSINPUTbaselinestretch interface can be used to increase the line
% spacing as well. Be sure and use the nomarkers option of endfloat to
% prevent endfloat from "marking" where the figures would have been placed
% in the text. The two hack lines of code above are a slight modification of
% that suggested by in the endfloat docs (section 8.4.1) to ensure that
% the full captions always appear in the list of figures/tables - even if
% the user used the short optional argument of \caption[]{}.
% IEEE papers do not typically make use of \caption[]'s optional argument,
% so this should not be an issue. A similar trick can be used to disable
% captions of packages such as subfig.sty that lack options to turn off
% the subcaptions:
% For subfig.sty:
% \let\MYorigsubfloat\subfloat
% \renewcommand{\subfloat}[2][\relax]{\MYorigsubfloat[]{#2}}
% However, the above trick will not work if both optional arguments of
% the \subfloat command are used. Furthermore, there needs to be a
% description of each subfigure *somewhere* and endfloat does not add
% subfigure captions to its list of figures. Thus, the best approach is to
% avoid the use of subfigure captions (many IEEE journals avoid them anyway)
% and instead reference/explain all the subfigures within the main caption.
% The latest version of endfloat.sty and its documentation can obtained at:
% http://www.ctan.org/tex-archive/macros/latex/contrib/endfloat/
%
% The IEEEtran \ifCLASSOPTIONcaptionsoff conditional can also be used
% later in the document, say, to conditionally put the References on a 
% page by themselves.




% *** PDF, URL AND HYPERLINK PACKAGES ***
%
\usepackage{url}
% url.sty was written by Donald Arseneau. It provides better support for
% handling and breaking URLs. url.sty is already installed on most LaTeX
% systems. The latest version and documentation can be obtained at:
% http://www.ctan.org/tex-archive/macros/latex/contrib/url/
% Basically, \url{my_url_here}.

\usepackage[utf8]{inputenc}
\usepackage[T1]{fontenc}
\usepackage[spanish]{babel}
\usepackage{graphicx}
\graphicspath{ {images/} }



% *** Do not adjust lengths that control margins, column widths, etc. ***
% *** Do not use packages that alter fonts (such as pslatex).         ***
% There should be no need to do such things with IEEEtran.cls V1.6 and later.
% (Unless specifically asked to do so by the journal or conference you plan
% to submit to, of course. )


% correct bad hyphenation here
\hyphenation{op-tical net-works semi-conduc-tor}


\begin{document}
%
% paper title
% Titles are generally capitalized except for words such as a, an, and, as,
% at, but, by, for, in, nor, of, on, or, the, to and up, which are usually
% not capitalized unless they are the first or last word of the title.
% Linebreaks \\ can be used within to get better formatting as desired.
% Do not put math or special symbols in the title.
\title{Bitcoin}
%
%
% author names and IEEE memberships
% note positions of commas and nonbreaking spaces ( ~ ) LaTeX will not break
% a structure at a ~ so this keeps an author's name from being broken across
% two lines.
% use \thanks{} to gain access to the first footnote area
% a separate \thanks must be used for each paragraph as LaTeX2e's \thanks
% was not built to handle multiple paragraphs
%
%
%\IEEEcompsocitemizethanks is a special \thanks that produces the bulleted
% lists the Computer Society journals use for "first footnote" author
% affiliations. Use \IEEEcompsocthanksitem which works much like \item
% for each affiliation group. When not in compsoc mode,
% \IEEEcompsocitemizethanks becomes like \thanks and
% \IEEEcompsocthanksitem becomes a line break with idention. This
% facilitates dual compilation, although admittedly the differences in the
% desired content of \author between the different types of papers makes a
% one-size-fits-all approach a daunting prospect. For instance, compsoc 
% journal papers have the author affiliations above the "Manuscript
% received ..."  text while in non-compsoc journals this is reversed. Sigh.



% note the % following the last \IEEEmembership and also \thanks - 
% these prevent an unwanted space from occurring between the last author name
% and the end of the author line. i.e., if you had this:
% 
% \author{....lastname \thanks{...} \thanks{...} }
%                     ^------------^------------^----Do not want these spaces!
%
% a space would be appended to the last name and could cause every name on that
% line to be shifted left slightly. This is one of those "LaTeX things". For
% instance, "\textbf{A} \textbf{B}" will typeset as "A B" not "AB". To get
% "AB" then you have to do: "\textbf{A}\textbf{B}"
% \thanks is no different in this regard, so shield the last } of each \thanks
% that ends a line with a % and do not let a space in before the next \thanks.
% Spaces after \IEEEmembership other than the last one are OK (and needed) as
% you are supposed to have spaces between the names. For what it is worth,
% this is a minor point as most people would not even notice if the said evil
% space somehow managed to creep in.

\author{\IEEEauthorblockN{Priscilla~Piedra y Martín~Flores}\\
        \IEEEauthorblockA{
        Escuela de Ingeniería en Computación\\
        Instituto Tecnológico de Costa Rica. Cartago, Costa Rica
        }\\
        \small{\texttt{\{ppiedra90, mfloresg\}}\texttt{@gmail.com}}% <-this % stops a space
\thanks{Este documento fue realizado durante el curso Redes de Computadoras Avanzadas, impartido por el profesor Luis Carlos Loaiza Canet. Programa de Maestría en Computación, Instituto Tecnológico de Costa Rica. Segundo Semestre, 2017.}
}

% The paper headers
%\markboth{Journal of \LaTeX\ Class Files,~Vol.~13, No.~9, September~2014}%
%{Shell \MakeLowercase{\textit{et al.}}: Bare Demo of IEEEtran.cls for Computer Society Journals}

\markboth{Redes de Computadoras Avanzadas, Noviembre 2017}%
{Shell \MakeLowercase{\textit{et al.}}: Bare Demo of IEEEtran.cls for Computer Society Journals}


% The only time the second header will appear is for the odd numbered pages
% after the title page when using the twoside option.
% 
% *** Note that you probably will NOT want to include the author's ***
% *** name in the headers of peer review papers.                   ***
% You can use \ifCLASSOPTIONpeerreview for conditional compilation here if
% you desire.



% The publisher's ID mark at the bottom of the page is less important with
% Computer Society journal papers as those publications place the marks
% outside of the main text columns and, therefore, unlike regular IEEE
% journals, the available text space is not reduced by their presence.
% If you want to put a publisher's ID mark on the page you can do it like
% this:
%\IEEEpubid{0000--0000/00\$00.00~\copyright~2014 IEEE}
% or like this to get the Computer Society new two part style.
%\IEEEpubid{\makebox[\columnwidth]{\hfill 0000--0000/00/\$00.00~\copyright~2014 IEEE}%
%\hspace{\columnsep}\makebox[\columnwidth]{Published by the IEEE Computer Society\hfill}}
% Remember, if you use this you must call \IEEEpubidadjcol in the second
% column for its text to clear the IEEEpubid mark (Computer Society jorunal
% papers don't need this extra clearance.)



% use for special paper notices
%\IEEEspecialpapernotice{(Invited Paper)}



% for Computer Society papers, we must declare the abstract and index terms
% PRIOR to the title within the \IEEEtitleabstractindextext IEEEtran
% command as these need to go into the title area created by \maketitle.
% As a general rule, do not put math, special symbols or citations
% in the abstract or keywords.
\IEEEtitleabstractindextext{%
\begin{abstract}
Bitcoin tiene mas de 1.000 millones de dolares en el mercado circulando pues, han resultado ser el centro de atención de varios usuarios alrededor del mundo. La moneda virtual ha demostrado ser parte de una revolución en la economía donde no es necesario realizar transacciones a través de terceros. El bitcoin funciona como un protocolo de envío y recepción de mensajes \emph{open source} además de una red punto-a-punto. Se basa en que no es necesario un organismo central o un intermediario para llevar el control entre dos personas pues la misma red se encarga de velar por que las transacciones sean veraces. 
\end{abstract}

% Note that keywords are not normally used for peerreview papers.
%\begin{IEEEkeywords}
%Computer Society, IEEEtran, journal, \LaTeX, paper, template.
%\end{IEEEkeywords}
}


% make the title area
\maketitle


% To allow for easy dual compilation without having to reenter the
% abstract/keywords data, the \IEEEtitleabstractindextext text will
% not be used in maketitle, but will appear (i.e., to be "transported")
% here as \IEEEdisplaynontitleabstractindextext when the compsoc 
% or transmag modes are not selected <OR> if conference mode is selected 
% - because all conference papers position the abstract like regular
% papers do.
\IEEEdisplaynontitleabstractindextext
% \IEEEdisplaynontitleabstractindextext has no effect when using
% compsoc or transmag under a non-conference mode.



% For peer review papers, you can put extra information on the cover
% page as needed:
% \ifCLASSOPTIONpeerreview
% \begin{center} \bfseries EDICS Category: 3-BBND \end{center}
% \fi
%
% For peerreview papers, this IEEEtran command inserts a page break and
% creates the second title. It will be ignored for other modes.
\IEEEpeerreviewmaketitle



\IEEEraisesectionheading{\section{Introducción}\label{sec:introduction}}
% Computer Society journal (but not conference!) papers do something unusual
% with the very first section heading (almost always called "Introduction").
% They place it ABOVE the main text! IEEEtran.cls does not automatically do
% this for you, but you can achieve this effect with the provided
% \IEEEraisesectionheading{} command. Note the need to keep any \label that
% is to refer to the section immediately after \section in the above as
% \IEEEraisesectionheading puts \section within a raised box.




% The very first letter is a 2 line initial drop letter followed
% by the rest of the first word in caps (small caps for compsoc).
% 
% form to use if the first word consists of a single letter:
% \IEEEPARstart{A}{demo} file is ....
% 
% form to use if you need the single drop letter followed by
% normal text (unknown if ever used by IEEE):
% \IEEEPARstart{A}{}demo file is ....
% 
% Some journals put the first two words in caps:
% \IEEEPARstart{T}{his demo} file is ....
% 
% Here we have the typical use of a "T" for an initial drop letter
% and "HIS" in caps to complete the first word.


\IEEEPARstart{B}{itcoin} es una colección de conceptos y tecnologías que forma la base de un ecosistema monetario digital. La divisas llamada \emph{bitcoin} se usa para almacenar y transmitir valor entro los participantes de una red bitcoin. Los usuario de bitcoin se comunican entre sí usando el protocolo bitcoin principalmente a través de Internet, aunque otras redes de transporte puede también ser usadas. Las tecnologías del protocolo bitcoin, disponibles como código abierto, puede ser ejecutadas en un amplio rango de dispositivos de computación, incluyendo computadoras portátiles y teléfonos inteligentes, haciendo que esta tecnología sea de fácil acceso.

Los usuarios pueden transferir bitcoins en una red para hacer básicamente cualquier cosa que puede ser hecha con divisas convencionales, incluyendo la compra y venta de bienes, envío de dinero o para crédito. El bitcoin puede ser comprado, vendido e intercambiado por otras divisas con tipos de cambio especializados. En cierto sentido, bitcoin es la forma perfecta de dinero para el Internet porque es rápida, segura y sin fronteras.

A diferencia de divisas tradicionales, los bitcoins son totalmente virtuales. No hay monedas físicas o monedas digitales como tal. Las monedas están implícitas en transacciones que transfieren valor desde un emisor hacia un destinatario. Los usuarios de bitcoin tienen llaves que les permite probar la propiedad de un bitcoin en una red bitcoin. Con estas llaves puede firmar transacciones para desbloquear/liberar el valor y gastarlo al trasferirlo a un nuevo dueño. Las llaves son usualmente almacenadas en un monedero digital en la computadora o dispositivo de cada usuario. La posesión de una llave que pueda firmar una transacción es sólo el prerequisito para gastar bitcoins, se da el control total a los usuarios.

El bitcoin es un sistema distribuido punto-a-punto. Como tal no hay un servidor ``central'' o un punto de control. Los bitcoins son creado a través de un proceso llamado minería (\emph{minning}) el cual involucra competir para encontrar soluciones a un problema matemático mientras se procesa la transacción bitcoin. Cualquier participante de la red bitcoin (por ejemplo, cualquiera usando un dispositivo que corre el conjunto total del protocol bitcoin) puede operar como minero, usado el poder de procesamiento de su computadora para verificar y guardar transacciones. Cada 10 minutos, en promedio, un minero bitcoin es capaza de validar las transacciones de los 10 minutos pasados y se le premia con un nuevo bitcoin. Esencialmente, la minería de bitcoins descentraliza la emisión de moneda y otras funciones de un banco central y reemplaza la necesidad de tener uno.

El protocolo bitcoin incluye algoritmos que regulan la función de minería a través de la red. La dificultad de la tarea de procesamiento que los mineros deben realizar se ajusta dinámicamente, así en promedio, alguien tiene éxito cada 10 minutos independientemente de cuántos mineros (y qué tanto procesamiento) están compitiendo en algún momento. Cada 4 años, el protocolo también reduce a la mitad la proporción en la que un nuevo bitcoin es creado, y limita el número total de bitcoin que podrían ser creado a un total fijo por debajo de 21 millones de monedas. El resultado es que el número de bitcoin en circulación sigue una curva predecible que se acercará a 21 millones para el año 2140. Debido a la disminución de la tasa de emisión de Bitcoin, a largo plazo, la moneda de Bitcoin es deflacionista. Además, bitcoin no puede ser inflada ``imprimiendo'' nueva moneda por encima o por debajo del índice esperado de emisión.

Detrás de escenas, bitcoin es también el nombre del protocolo, una red punto-a-punto. La divisa bitcoin es realmente sólo la primer aplicación de esta invención. Bitcoin representa la culminación de décadas de investigación en criptografía y sistemas distribuidos e incluye cuatro innovaciones claves que se unen de una forma única y poderosa. Bitcoin consiste de:
\begin{itemize}
    \item Una red descentralizada punto-a-punto (el protocolo bitcoin)
    \item Una libro de transacciones públicas (blockchain)
    \item Un conjunto de reglas para validación de transacciones y emisión de divisas (reglas de concenso)
    \item Un mecanismo para alcanzar consenso descentralizado global en un blockchain válido (algoritmo de prueba de trabajo\footnote{\emph{Proof-of-Work Algorithm}})
\end{itemize}

\subsection{Historia}
Bitcoin fue inventado en el 2008 con la publicación de un artículo titulado \emph{``Bitcoin: A Peer-to-Peer Electronic Cash System``} escrito bajo el álias de Satoshi Nakamoto. Nakamoto combinó varias invenciones previas como b-money y HashCash para crear un sistema de efectivo electrónico completamente descentralizado que no depende de una autoridad central para emisión de divisas o liquidación y validación de transacciones. La innovación clave fue usar un sistema de computación distribuida (llamado el algoritmo de prueba de trabajo) para conducir una ``elección'' global cada 10 minutos, permitiendo a la red descentralizada lograr un consenso acerca el estado de transacciones. Esto resuelve elegantemente el problema del doble-gasto en donde una sola unidad en una divisa podía ser gastada dos veces. Previamente, el problema del doble-gasto fue una debilidad de las divisas digitales y fue abordado limpiando todas las transacciones a través de una ``casa de limpieza'' central.

La red bitcoin inició en el 2009, basado en una implementacio4n de referencia publicada por Nakamoto y desde ese momento revisada por muchos otros programadadores. La implementación del algoritmo de prueba de trabajo (minería) que proporciona seguridad y resiliencia para bitcoin ha incrementado su poder exponencialmente, y ahora excede la poder de procesamiento combinado de las principales supercomputadoras del mundo. El valor total de mercado de Bitcoin ha excedido en ocasiones los \$35 millones, dependiendo de la tasa de intercambio bitcoin-a-dólar. Las transacción más grande procesada hasta ahora por la red fue \$150 millones transmitidos instantaneamente y procesada sin ninguna comisión.

Satoshi Nakamoto se retiró del público en abril del 2011, dejando la responsabilidad de desarrollo del código y la red a un grupo de voluntarios. La identidad de la persona o personas detrás de bitcoin aún se desconoce. Sin embargo, ni Satoshi Nakamoto ni alguien más ejerce control individual sobre el sistema bitcoin, el cual opera basado en principios matemáticos transparentes, código abierto y consenso entre los participantes. La invención como tal es pionera y se ha engendrado nueva ciencia en los campos de computación distribuida y economía. 

\subsection{Iniciando con Bitcoin}
Bitcoin es un protocolo que puede ser accesado usando un aplicación cliente que hable el protocolo. Un ``monedero bitcoin'' es la interfaz de usuario más común con el sistema bitcoin. Existen varias implementaciones y marcas de monederos bitcoin, varían en calidad, rendimiento, seguridad, privacidad y confiabilidad. Está también la implementación de referencia del protocolo bitcoin que incluye un monedero conocido como el ``Satoshi Client'' o ``Bitcoin Core'', que se deriva de la implementación original escrita por Satoshi Nakamoto.

\subsubsection{Obteniendo el primer Bitcoin}
La primer y más difícil tarea para los nuevos usuarios es adquirir algún bitcoin. A diferencia de otras divisas extranjeras, no se puede comprar aún bitcoins en un banco o en un kiosko de intercambio de divisas.

Las transacciones de bitcoins son irreversibles. La mayoría de redes de pago como tarjetas de crédito, débito, PayPal y cuentas de banco son reversibles. Para alguien que vende bitcoin, esta diferencia introduce un riesgo muy alto de que el comprador revierta el pago electrónico después de haber recibido bitcoin, en efecto defraudando al vendedor. Para mitigar este riesgo, compañías que aceptan pagos electrónicos tradicionales en retorno de bitcoin usualmente le solicitan a los compradores pasar por verificaciones de identidad y crédito que puede tomar días o semanas. Como nuevo usuario, esto significa que no se puede comprar bitcoin instantaneamente con una tarjeta de crédito. Algunos métodos para obtener bitcoins como nuevo usuario son:
\begin{itemize}
    \item Buscar un amigo que tenga bitcoin y comprarle algunos directamente.
    \item Usar un servicio clasificado como \url{localbitcoins.com} para encontrar vendedores y comprar bitcoin en una transacción en persona.
    \item Se puede ganar bitcoins por medio de la venta de un producto o servicio.
    \item Utilizar un ATM de bitcoin que acepte efectivo y envíe bitcoins a un monedero. 
\end{itemize}

\section{¿Cómo funciona Bitcoin?}
El sistema bitcoin, a diferencia de sistemas de banca y pagos tradicionales, está basado en confianza descentralizada. En lugar de una autoridad central de confianza, en bitcoin, la confianza se logra como una propiedad emergente de las interacciones de diferentes participantes en el sitema bitcoin. 

En la figura \ref{fig:bitcoin-overview} se ve que el sistema bitcoin conside de usuarios con monederos que contienen llaves, transacciones que son propagadas a través de la red y mineros que producen (a través de computación competitiva) el \emph{consensus blockchain}, que es el libro de mayor autoridad de todas las transacciones. 

\begin{figure*}[h]
    \center
    \includegraphics[width=15cm]{bitcoin-overview}
    \caption{Bitcoin: visión general}
    \label{fig:bitcoin-overview}
\end{figure*}

\subsection{Transacciones Bitcoin}
En términos simples, una transacción le dice a la red que el propietario de algún valor de bitcoin ha autorizado la transferencia de ese valor a otro propietario. El nuevo propietario puede ahora gasta el bitocoin creando una nueva transacción que autoriza transferir a otro propietario y así sucesivamente, en una cadena de propiedad.

\subsubsection{Entradas y Salidas de la Transacción}

Las transacciones son como líneas en un libro de contabilidad de doble entrada \footnote{\emph{Double-entry bookkeeping ledger}}. Cada transacción contiene una o más ``entradas'', que son como débitos contra una cuenta bitcoin. En el otro lado de la trasacción, hay una o más ``salidas'', que son como créditos agregados a una cuenta bitcoin. Las entradas y salidas (débitos y créditos) no necesariamente suman la misma cantidad. Las salidas se suman a un poco menos que las entradas y la diferencia representa una tarifa de transacción implícita, que es un pequeño pago cobrado por el minero que incluye la transacción en el libro mayor. Una transacción bitcoin se muestra como un libro de contailidad en la figura \ref{fig:bookkepping}.

La transacción también contiene pruebas de propiedad para cada cantidad de bitcoin (entradas) cuyo valor se está gastando, en forma de una firma digital del propietario, que puede ser independientemente validado por cualquiera. En términos bitcoin, ``gastar'' es firmar una transacción que transfiere el valor de una transacción anterior a un nuevo propietario identificado por una dirección de bitcoin.

\begin{figure}[h]
    \center
    \includegraphics[width=9cm]{bookkepping}
    \caption{Transacción Bitcoin vista como un libro de contabilidad.}
    \label{fig:bookkepping}
\end{figure}

\subsubsection{Cadenas de transacciones}
Un pago usa la salida una transacción previa como su entrada. Las transacciones forman una cadena, en donde las entras de la última transacción corresponden a la salida de transacciones previas. La llave de un usuario proporciona la firma que libera esas transacciones previas, probando así a la red bitcoin que el usuario es dueño de los fondos. Una cadena de transacciones se muestra en la figura \ref{fig:chain}

\begin{figure}[h]
    \center
    \includegraphics[width=9cm]{chain}
    \caption{Una cadena de transacciones, donde la salida de una transacción es la entrada de la otra.}
    \label{fig:chain}
\end{figure}

Muchas transacciones bitcoin incluyen salidas que referencias ambos, una dirección del nuevo propietario y una dirección del propietario actual, llamada la \emph{change address}. Esto es porque las entradas de la transacción no pueden ser divididas. Si se compra un artículo de \$5 en una tienda pero se usa un billete de \$20 para pagarlo, se espera recibir \$15 como cambio. El mismo concepto aplica con las entradas de transacción en bitcoin. Si se compra un artículo que cuesta 5 bitcoins pero solo se tiene 20 bitcoins para usar, se enviaría una salida de 5 bitcoin al propietario de la tienda y una salida de 15 bitcoin de vuelta al usuario como cambio (menos cualquier comisión de transacción aplicable). El \emph{change address} no tiene que tener la misma dirección que la que se usó como entrada y por razones de privacidad es usualmente una nueva dirección del monedero del propietario.

En resumen, las transacciones mueven valor desde entradas de transacción a salidas de transacción. Una entrada es una referencia a una salida de transacción previa, que muestra de donde viene el valor. Una salida de transacción direcciona un valor específico a una nueva dirección de propietario bitcoin y puede incluir una salida de cambio de vuelta al propietario original. Las salidas de una transacción pueden ser usadas como entradas en una nueva transacción, creando así una cadena de propiedad conforme el nuevo valor se mueve de un propietario a otro. 

\subsection{Formas comunes de transacción}
La forma más comurn de transacción es un pago simple desde una dirección a otra, la cual usualmente incluye algun ``cambio'' retornado al propietario original. Este tipo de transacción tiene una entrada y dos salidas como se muestra en la figura \ref{fig:trans1}    

\begin{figure}[h]
    \center
    \includegraphics[width=9cm]{trans1}
    \caption{La transacción más común}
    \label{fig:trans1}
\end{figure}

Otra forma común de transacción es una que agrega varias entradas en una sola salida. Esto representa el equivalente en el mundo real a cambiar una pila de monedas y billetes por una sola nota. Transacciones como estas son algunas veces generadas por aplicaciones de monederos para limpiar muchos pequeños montos que fueron recibidos como cambio de pagos previos.

\begin{figure}[h]
    \center
    \includegraphics[width=9cm]{trans2}
    \caption{Transacción de fondos agregados}
    \label{fig:trans2}
\end{figure}

Finalmente, otra forma de transacción que se ve con frecuencia en el libro de bitcoin es una transacción que distribuye una entrada en varias salidas representando múltiples destinatarios. Este tipo de transacción la usan a veces las entidades comerciales para distribuir fondos, como cuando se procesa el pago de la nómina a múltiples empleados.

\begin{figure}[h]
    \center
    \includegraphics[width=9cm]{trans3}
    \caption{Transacción de fondos distribuidos}
    \label{fig:trans3}
\end{figure}

\subsection{Construyendo una Transacción}
La aplicación de monedero de un usuario contiene toda la lógica para seleccionar las entradas y salidas apropiadas para construir una transacción. El usuario sólo necesita especificar un destino y un monto y el resto sucede en la aplicación de monedero sin que se vean los detalles. La aplicación de monedero puede construir transacciones inclusive cuando esta totalmente fuera de línea. Tal y como escribir un cheque en la casa y enviarlo luego al banco, la transacción no necesita que se construya y se firme mientras se está conectado a la red bitcoin.

\subsubsection{Obteniendo las entradas apropiadas}
La aplicación de monedero primero tiene que buscar las entradas para que pueda pagar la cantidad deseada. La mayoría de los monederos llevan un registro de todas las salidas disponibles que pertenecen a la dirección del monedero. Una aplicación de monedero que se ejecute como \emph{full-node client} contiene una copia de todas las salidas pendientes para cualquier transacción en el blockchain. Esto permite que el monedero construya entradas de transacciones y verifique rápidamente que las transacciones entrantes tengan entradas correctas. Sin embargo, debido a que un \emph{full-client node} ocupa mucho espacio en disco, la mayoría de los usuarios de monederos usan la versión ``liviana'' que solo registra las salidas no utilizadas del usuario.

Si la aplicación de monedero no mantiene una copia de las transacciones pendientes, puede entonces hacer una consulta a la red bitcoin para obtener esta información usando una variedad de API\footnote{\emph{Application Programming Interface}} disponibles por diferentes proveedores o preguntando a un \emph{full-node} usando una llamada por programación.

\subsubsection{Creando las salidas}
Una salida de transacción se crea en forma de un script que crea un gravamen en el valor y solo se puede canjear mediante la introducción de una solución al script. En términos simples, la salida de la transacción de un usuario $A$ va contener un script que dice algo como ``Esta salida se paga a quién pueda presentar una firma desde la llave correspondiente a la dirección pública de $B$, siendo $B$ en este caso el destinatario de la transacción. Debido a que sólo el usuario $B$ tiene el monedero con las llaves correspondientes a esa dirección, solamente el monedero de $B$ puede presentar dicha firma para redimir esta salida. Por lo tanto, $A$ ``gravará'' el valor de la salida con una demanda de una firma de $B$.

Finalmente, para que la transacción pueda ser procesada por la red de manera oportuna, la aplicación de monedero de $A$ agregará una pequeña comisión. La comisión de transacción que colecta el minero es una tarifa para validar e incluir la transacción en un bloque que se registrará en el blockchain.

\subsubsection{Agregando la transacción al libro}
La transacción anteriormente creada contiene ahora todo lo necesario para confirmar la origen de los fondos y asignar nuevos propietarios. Ahora la transacción debe ser transmitida a la red bitcoin donde será parte del blockchain. 

\paragraph{Transmitiendo la transacción} debido a que la transacción contiene toda la información necesario para procesarse, no importa cómo o dónde se transmite en la red bitcoin. La red bitcoin es una red punto-a-punto, en donde cada cliente bitcoin participa al conectarse con otros clientes bitcoin. El propósito de la red bitcoin es propagar transacciones y bloques a todos los participantes.

\paragraph{¿Cómo se propaga?} Cualquier sistema, como un servidor, aplicación o monedero que participa en la red bitcoin al ``hablar'' el protocolo bitcoin se llama un nodo bitcoin. La aplicación de monedero de un usuario puede enviar la nueva transacción a cualquier nodo bitcoin. El monedero no necesita estar conectado directamente al monedero de otro usuario tampoco. Cualquier nodo bitcoin que reciba una transacción válida que no se haya visto antes lo reenviará inmediatamente a todos los otros nodos a los que esté conectado, una técnica de propagación conocida como inundación (\emph{flooding}). Así, la transacción se propaga rápidamente a través de la red punto-a-punto, alcanzando un gran porcentaje de los nodos en unos pocos segundos.

\section{Minando Bitcoin}
Una transacción que se propaga a través de la red bitcoin no se convierte en parte del blockchain hasta que sea verificada e incluida en un bloque por un proceso llamado minería (\emph{minning}).

El sistema de confianza de bitcoin está basado en computación. Las transacciones se agrupan en bloques, que requieren una gran cantidad de cómputo para probar, pero solo una pequeña cantidad de cómputo para verificarlo como comprobado. El proceso de minería tiene dos propósitos en bitcoin:
\begin{itemize}
    \item Los nodos de minería validan todas las transacciones con referencia en las reglas de consenso de bitcoin. Por lo tanto, la minería proporciona seguridad para las transacciones bitcoin al rechazar transacciones inválidas o mal formadas.
    \item La minería crea un nuevo bitcoin en cada bloque, casi como un banco central imprime nueva moneda. La cantidad de bitcoin creado por bloque es limitado y disminuye con el tiempo, siguiendo un calendario de emisión fijo.  
\end{itemize}

La minería logra un fino equilibrio entre los costos y la reconmpensa. La minería usa electricidad para resolver un problema matemático. Un minero exitoso va a recolectar una recompensa en la forma de nuevos bitcoins y comisiones de transacciones. Sin embargo, la recompensa solamente será recolectada si un minero ha validado correctamente todas las transacciones, para satisfacer las reglas de concenso. Este delicado balance proporciona seguridad para bitcoin sin una autoridad central.

Una buena forma para describir la minería es como un gigante juego competitivo de sudoku que se reinicia cada vez que alguien encuentra una solución y cuya dificultad automáticamente se ajusta para que tome aproximadamente 10 minutos para encontrar la solución. Este juego gigante de sudoku se puede imaginar con un tamaño de miles de columnas y filas. Si se muestra que el juego fue resuelto se puede verificar rápidamente. Sin embargo, si el juego tiene varios cuadros llenos y varias vacíos, tomará mucho trabajo en ser resuelto. La dificultad del sudoku puede ser ajustada al cambiar su tamaño (más o menos filas y columnas), pero aún asi puede ser verificado fácilmente. El ``juego'' usado en bitcoin se base en una función \emph{hash} criptográfica y expone características similares: es asimétricamente difícil de resolver pero fácil de verificar, y esta dificultad puede ser ajustada.

\section{Minando Transacciones en Bloques}
Nuevas transacciones están constantemente fluyendo dentro de la red desde monederos de usuarios y otras aplicaciones. Como estos son vistos por la red bitcoin como nodos, se agregan a un \emph{pool} temporal de transacciones no verificadas mantenidas por cada nodo. Conforme los mineros construyen un nuevo bloque, van agregando transacciones no verificadas desde este \emph{pool} a un nuevo bloque y luego intentan probar la validez de ese nuevo bloque, con el algoritmo de minería de prueba-de-trabajo. 

Las transacciones son agregadas a un nuevo bloque, priorizadas de acuerdo con las comisiones de transacción más altas y alguna otros criterios. Cada minero inicia el proceso de minería de un nuevo bloque de transacciones tan pronto y como recibe un bloque previo de la red, a sabiendas que a perdido la ronda previa de competición. Inmediatamente crea un nuevo bloque, lo llena con las transacciones y una huella (\emph{fingerprint}) del bloque previo, e inicia el cálculo de la prueba-de-trabajo para el nuevo bloque. Canda minero incluye una transacción especial a su bloque, uno que paga su propia dirección de bitcoin la recompensa del bloque más la suma de comisiones de trasacción de todas las transacciones incluidas en el bloque. Si encuentra una solución que haga que ese bloque sea válido, él ``gana'' su recompensa porque su bloque exitoso se agregó al blockchain global y la transacción de recompensa que él incluyó se vuelve en gastable.

\subsection{Ejemplo: Jing y Alice}
Jing es un usuario que lleva varios años participando del proceso de minería. Ha configurado su software para crear nuevos bloques que asignan un premio al \emph{pool} de direcciones. A partir de ahí, una parte del premio es distribuido a Jing y a otros mineros en proporción a la cantidad de trabajo con el que contribuyan en la última ronda.

Alice ha realizado una transacción y esta ha sido tomada por la red e incluida en un \emph{pool} de transacciones no verificadas. Una vez validada por el software de minería fue incluida en un nuevo bloque, llamado bloque candidato, generado por el \emph{pool} de minería de Jing. Todos los mineros participando en ese \emph{pool} de minería inician inmediatamente el cálculo de la prueba-de-trabajo para el bloque candidato. Aproximadamente cinco minutos después que la transacción fue transmitida por el monedero de Alice, uno de los circuitos de minería de Jing encuentra la solución para el bloque candidato y lo anuncia a la red. Mientras que los otros mineros valida el bloque ganador, ellos inician la carrera para generar el próximo bloque.


El bloque ganador de Jing se vuelve parte del blockchain como el bloque \#277316, que contiene 420 transacciones, incluyendo la transacción de Alice. El bloque que contiene la transacción de Alice se cuenta como una ``confirmación'' de esa transacción.

Aproximadamente 19 minutos después, un nuevo bloque \#277317, es minado por otro minero. Debido a que este nuevo bloque está construido por encima del bloque \#277316 que contiene la transacción de Alice, se añade aun más computación al blockchain, por lo tanto fortalece la confianza en esas transacciones. Cada bloque minado sobre el que contiene la transacción cuenta como una confirmación adicional para la transacción de Alicia. Conforme los bloques se apilan uno por encima del otro, se vuelve exponencialmente más difícil reversar la transacción, de esta forma se hace más y más confiable por la red.

En la figura \ref{fig:block}, se puede ver el bloque \#277316, que contiene la transacción de Alice. Por debajo hay 277316 bloques (incluyendo el bloque \#0), enlazados unos con otros en una cadena de bloques (blockchain) hasta regresar al bloque \#0, conocido como el bloque \emph{genesis}. Con el tiempo, conforme la ``altura'' en bloques aumenta, también lo hace la dificultad de calcular cada bloque y la cadena como un todo. Los bloques minados luego del que contiene la transacción de Alice actúan como una garantía adicional, ya que acumulan más cómputos en una cadena cada vez más larga. Por convención cualquier bloque con más de seis confirmaciones es considerado irrevocable, porque va a requerir una inmensa cantidad de computación para invalidar y recalcular seis bloques.

\begin{figure}[h]
    \center
    \includegraphics[width=9cm]{block}
    \caption{Transacción de fondos agregados}
    \label{fig:block}
\end{figure}

\subsection{Gastando la Transacción}
Ahora que la transacción de Alice ha sido agregada en el blockchain como parte de un bloque, es parte de un libro distribuido de bitcon y es visitble a todas las aplicaciones bitcoin. Cada cliente bitcoin puede verificar independientemente la transacción y considerarla válida para gastar. Los clientes \emph{full-node} pueden rastrear el origen de los fondos desde el momento en que el bitcoin fue generado en un bloque, de forma incremental, transacción a transacción, hasta que llegue a la dirección de un destinatario. Los clientes ``livianos'' pueden hacer lo que se llama un verificación de pago simplificada al confirmar que la transacción está en el blockchain y tiene varias bloques minados despues de ella, lo que garantiza que los mineros la aceptaron como válida.

El destinatario de los fondos puede ahora gastar la salida de la transacción y conforme recibe pagos de otros, él a su vez extiende la cadena de transacciones. Un ejemplo de esta cadena de transacciones agregadas se puede ver en la figura \ref{fig:transaction-chain}

\begin{figure}[h]
    \center
    \includegraphics[width=9cm]{transaction-chain}
    \caption{Las transacciones de Alice como parte de una cadena de transacciones de Joe hasta Gopesh}
    \label{fig:transaction-chain}
\end{figure}

% needed in second column of first page if using \IEEEpubid
%\IEEEpubidadjcol


% An example of a floating figure using the graphicx package.
% Note that \label must occur AFTER (or within) \caption.
% For figures, \caption should occur after the \includegraphics.
% Note that IEEEtran v1.7 and later has special internal code that
% is designed to preserve the operation of \label within \caption
% even when the captionsoff option is in effect. However, because
% of issues like this, it may be the safest practice to put all your
% \label just after \caption rather than within \caption{}.
%
% Reminder: the "draftcls" or "draftclsnofoot", not "draft", class
% option should be used if it is desired that the figures are to be
% displayed while in draft mode.
%
%\begin{figure}[!t]
%\centering
%\includegraphics[width=2.5in]{myfigure}
% where an .eps filename suffix will be assumed under latex, 
% and a .pdf suffix will be assumed for pdflatex; or what has been declared
% via \DeclareGraphicsExtensions.
%\caption{Simulation results for the network.}
%\label{fig_sim}
%\end{figure}

% Note that IEEE typically puts floats only at the top, even when this
% results in a large percentage of a column being occupied by floats.
% However, the Computer Society has been known to put floats at the bottom.


% An example of a double column floating figure using two subfigures.
% (The subfig.sty package must be loaded for this to work.)
% The subfigure \label commands are set within each subfloat command,
% and the \label for the overall figure must come after \caption.
% \hfil is used as a separator to get equal spacing.
% Watch out that the combined width of all the subfigures on a 
% line do not exceed the text width or a line break will occur.
%
%\begin{figure*}[!t]
%\centering
%\subfloat[Case I]{\includegraphics[width=2.5in]{box}%
%\label{fig_first_case}}
%\hfil
%\subfloat[Case II]{\includegraphics[width=2.5in]{box}%
%\label{fig_second_case}}
%\caption{Simulation results for the network.}
%\label{fig_sim}
%\end{figure*}
%
% Note that often IEEE papers with subfigures do not employ subfigure
% captions (using the optional argument to \subfloat[]), but instead will
% reference/describe all of them (a), (b), etc., within the main caption.
% Be aware that for subfig.sty to generate the (a), (b), etc., subfigure
% labels, the optional argument to \subfloat must be present. If a
% subcaption is not desired, just leave its contents blank,
% e.g., \subfloat[].


% An example of a floating table. Note that, for IEEE style tables, the
% \caption command should come BEFORE the table and, given that table
% captions serve much like titles, are usually capitalized except for words
% such as a, an, and, as, at, but, by, for, in, nor, of, on, or, the, to
% and up, which are usually not capitalized unless they are the first or
% last word of the caption. Table text will default to \footnotesize as
% IEEE normally uses this smaller font for tables.
% The \label must come after \caption as always.
%
%\begin{table}[!t]
%% increase table row spacing, adjust to taste
%\renewcommand{\arraystretch}{1.3}
% if using array.sty, it might be a good idea to tweak the value of
% \extrarowheight as needed to properly center the text within the cells
%\caption{An Example of a Table}
%\label{table_example}
%\centering
%% Some packages, such as MDW tools, offer better commands for making tables
%% than the plain LaTeX2e tabular which is used here.
%\begin{tabular}{|c||c|}
%\hline
%One & Two\\
%\hline
%Three & Four\\
%\hline
%\end{tabular}
%\end{table}


% Note that the IEEE does not put floats in the very first column
% - or typically anywhere on the first page for that matter. Also,
% in-text middle ("here") positioning is typically not used, but it
% is allowed and encouraged for Computer Society conferences (but
% not Computer Society journals). Most IEEE journals/conferences use
% top floats exclusively. 
% Note that, LaTeX2e, unlike IEEE journals/conferences, places
% footnotes above bottom floats. This can be corrected via the
% \fnbelowfloat command of the stfloats package.




\section{Conclusión}

El bitcoin puede ser útil para mucha gente pues es una moneda internacional además, puede ser usado en cualquier país sin tener que convertir monedas. Blockchain es realmente seguro y permite asegurar las transacciones del bitcoin, también las personas que reciben bitcoins no tendrán que pagar nada por las transacciones. El bitcoin presenta  algunas desventajas, pero algunas de ellas se deben a que es un concepto nuevo, así que a medida que pase el tiempo serán mas populares como método de pago.





% if have a single appendix:
%\appendix[Proof of the Zonklar Equations]
% or
%\appendix  % for no appendix heading
% do not use \section anymore after \appendix, only \section*
% is possibly needed

% use appendices with more than one appendix
% then use \section to start each appendix
% you must declare a \section before using any
% \subsection or using \label (\appendices by itself
% starts a section numbered zero.)
%


%\appendices
%\section{Proof of the First Zonklar Equation}
%Appendix one text goes here.
%
%\section{}
%Appendix two text goes here.


% use section* for acknowledgment
%\ifCLASSOPTIONcompsoc
%  % The Computer Society usually uses the plural form
%  \section*{Acknowledgments}
%\else
%  % regular IEEE prefers the singular form
%  \section*{Acknowledgment}
%\fi


%The authors would like to thank...


% Can use something like this to put references on a page
% by themselves when using endfloat and the captionsoff option.
\ifCLASSOPTIONcaptionsoff
  \newpage
\fi



% trigger a \newpage just before the given reference
% number - used to balance the columns on the last page
% adjust value as needed - may need to be readjusted if
% the document is modified later
%\IEEEtriggeratref{8}
% The "triggered" command can be changed if desired:
%\IEEEtriggercmd{\enlargethispage{-5in}}

% references section

% can use a bibliography generated by BibTeX as a .bbl file
% BibTeX documentation can be easily obtained at:
% http://www.ctan.org/tex-archive/biblio/bibtex/contrib/doc/
% The IEEEtran BibTeX style support page is at:
% http://www.michaelshell.org/tex/ieeetran/bibtex/
%\bibliographystyle{IEEEtran}
% argument is your BibTeX string definitions and bibliography database(s)
%\bibliography{IEEEabrv,../bib/paper}
%
% <OR> manually copy in the resultant .bbl file
% set second argument of \begin to the number of references
% (used to reserve space for the reference number labels box)
\begin{thebibliography}{1}

\bibitem{nakamoto}
S.~Nakamoto. \emph{Bitcoin:A Peer-to-Peer Electronic Cash System}. \hskip 1em plus 0.5em minus 0.4em\relax Que. www.bitcoin.org

\bibitem{}
A M.~Antonopoulos. \emph{Mastering Bitcoin} \hskip 1em plus 0.5em minus 0.4em\relax. Segunda Edición. O'Reilly Media, Inc. ISBN: 9781491954386. Junio 2017.


\end{thebibliography}

% biography section
% 
% If you have an EPS/PDF photo (graphicx package needed) extra braces are
% needed around the contents of the optional argument to biography to prevent
% the LaTeX parser from getting confused when it sees the complicated
% \includegraphics command within an optional argument. (You could create
% your own custom macro containing the \includegraphics command to make things
% simpler here.)
%\begin{IEEEbiography}[{\includegraphics[width=1in,height=1.25in,clip,keepaspectratio]{mshell}}]{Michael Shell}
% or if you just want to reserve a space for a photo:

\begin{IEEEbiography}[{\includegraphics[width=1in,height=1.25in,clip,keepaspectratio]{priscilla-piedra}}]{Priscilla Piedra}
es Ingeniera de Computación del Tecnologíco de Costa Rica. Actualmente es estudiante del programa de Maestría en Ciencas de la Computación en la misma universidad. Sus principales intereses son: \emph{cloud computing} y automatización. 
\end{IEEEbiography}

% if you will not have a photo at all:
\begin{IEEEbiography}[{\includegraphics[width=1in,height=1.25in,clip,keepaspectratio]{martin-flores}}]{Martín Flores}
es Ingeniero en Informática de la Universidad Nacional. Actualmente, realiza sus estudios de Maestría en Ciencias de la Computación del Tecnológico de Costa Rica. Sus principales intereses son: lenguajes de programación, ingeniería de software y \emph{DevOps}.
\end{IEEEbiography}

% insert where needed to balance the two columns on the last page with
% biographies
%\newpage

%\begin{IEEEbiographynophoto}{Jane Doe}
%Biography text here.
%\end{IEEEbiographynophoto}

% You can push biographies down or up by placing
% a \vfill before or after them. The appropriate
% use of \vfill depends on what kind of text is
% on the last page and whether or not the columns
% are being equalized.

\vfill

% Can be used to pull up biographies so that the bottom of the last one
% is flush with the other column.
%\enlargethispage{-5in}



% that's all folks
\end{document}

